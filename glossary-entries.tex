%   Copyright (C) 2021 Greenweaves Software Limited
%
%   This program is free software: you can redistribute it and/or
%   modify it under the terms of the GNU Lesser General Public
%   License as published by the Free Software Foundation; either
%   version 2.1 of the License, or (at your option) any later version.
%
%  This program is distributed in the hope that it will be useful,
%  but WITHOUT ANY WARRANTY; without even the implied warranty of
%  MERCHANTABILITY or FITNESS FOR A PARTICULAR PURPOSE.  See the
%  GNU  Lesser General Public License for more details
%
%  You should have received a copy of the GNU Lesser General Public License
%  along with this program.  If not, see <https://www.gnu.org/licenses/>.
%
%  To contact me, Simon Crase, email simon@greenweaves.nz

\newacronym{ATP}{ATP}{adenosine $5\prime$-triphosphate}

\newacronym{ER}{ER}{Endoplasmic reticulum}

\newacronym{mAP}{mAP}{mean average precision\cite{hio2018mAP,mobassir2020understanding}}

\newacronym{MTOC}{MTOC}{microtubule-organizing centre}

\newacronym{ROI}{RO}{region of interest\cite{kumar2019instance}}

\newacronym{SPP}{SPP}{Spatial Pyramid Pooling}

\newglossaryentry{actin}{
	name={actin},
	description={Abundant structural protein in eukaryotic cells that interacts with many othe proteins. The monomeric globular form polymeryzes to form actin ilaments\cite{lodish2003molecular}}}


\newglossaryentry{aggresome} {
	name={aggresome},
	description={An aggresome is an aggregation of misfolded proteins in the cell, formed when the protein degradation system of the cell is overwhelmed\cite{enwiki:994315898}}}


\newglossaryentry{centrosome} {
	name= {centrosome},
	description={Structure located near the nucleus of animal and plant cells that is the primary \gls{MTOC}; in animals it contains a pair of centrioles embedded in a protein matrix\cite{lodish2003molecular}}}

\newglossaryentry{cytoplasm}{
	name= {cytoplasm},
	description={Viscous contents of a cell that are contained within the plasma membrane but, in eukaryotic cells, outside the nucleus\cite{lodish2003molecular}}}

\newglossaryentry{cytosol}{
	name= {cytosol},
	description={Unstructured aqueous phase of the \gls{cytoplasm} excluding organelles, membranes, and insoluble cytoskeletal  components\cite{lodish2003molecular}}}

\newglossaryentry{endoplasmic_reticulum}{
	name= {endoplasmic reticulum},
	description={Network of interconnected membranous structures within the \gls{cytoplasm} of eukaryotic cells continguous with the outer nuclear envelope\cite{lodish2003molecular}}}


\newglossaryentry{golgi_apparatus} {
	name={Golgi apparatus},
	description={Stacks of flattened, interconnected membrane-bounded compartments in eukaryotic cells that function in processing and sorting of proteins and lipids destined for other cellular components or for secretion.\cite{lodish2003molecular}}}

\newglossaryentry{intermediate_filament}{
	name={intermediate filament},
	description={Cytoskeletal fiber formed by polymerization of several types of 	subunit proteins including kertins, lamins, and vimentin\cite{lodish2003molecular}}}


\newglossaryentry{microtubules} {
	name= {microtubules} ,
	description={Cytoskeletal fiber that is formed by polymerization of $\alpha,\beta$-tubulin monomers, and exhibits structural and functional polarity. Microtubes are important components of cilia, flagella, the \gls{mitotic_spindle}, and other cellular structures\cite{lodish2003molecular}}}


\newglossaryentry{mitochondrion} {
	name= {mitochondrion},
	plural={mitochondria},
	description={Large \gls{organelle} that is surrounded by two phosphlipid bilayer membranes, contyains DNA, and carries out osidative phosphorylation, thereby producting much of the \gls{ATP} in eukaryotic cells\cite{lodish2003molecular} }}


\newglossaryentry{mitotic_spindle} {
	name={mitotic spindle},
	description={The mitotic spindle is a microtubule-made machine required for chromosome segregation during mitosis. Several pathways and molecules contribute to the assembly of this structure, which shows intrinsic dynamic properties\cite{bradshaw2015encyclopedia}}}

\newglossaryentry{nuclear_bodies} {
	name= {nuclear bodies},
	description={Nuclear bodies including nucleoli, Cajal bodies, nuclear speckles, Polycomb bodies, and paraspeckles are membrane-less subnuclear organelles. They are steady-state structures that dynamically respond to basic physiological processes as well as various forms of stress, altered metabolic conditions and alterations in cellular signaling\cite{mao2011biogenesis}}}

\newglossaryentry{nuclear_membrane}{
	name={nuclear membrane},
	description={A nuclear membrane is a double membrane that encloses the cell nucleus\cite{segre2021}}}

\newglossaryentry{nuclear_speckles}{
	name={nuclear speckles},
	description={Nuclear speckles, also known as interchromatin granule clusters, are nuclear domains enriched in pre-mRNA splicing factors, located in the interchromatin regions of the nucleoplasm of mammalian cells. When observed by immunofluorescence microscopy, they usually appear as 20–50 irregularly shaped structures that vary in size\cite{spector2011nuclear}}}

\newglossaryentry{nucleolus} {
	name= {nucleolus},
	plural={Nucleoli},
	description={Large structure in the nucleus of eukaryotic cells where rRNA synthesis and processing occurs and ribosome subunits are assembled\cite{lodish2003molecular}}}

\newglossaryentry{nucleolus_fibrillar_center}{
	name={nucleolus fibrillar center},
	description={The nucleolus fibrillar center (FC) is a sub-compartment of most metazoan nucleoli.\cite{nucleolus:fibrillar:center}}}

\newglossaryentry{nucleoplasm} {
	name={nucleoplasm},
	description={Similar to the cytoplasm of a cell, the nucleus contains nucleoplasm. The nucleoplasm is a type of protoplasm, and is enveloped by the nuclear envelope (also known as the nuclear membrane). The nucleoplasm includes the chromosomes and nucleolus.\cite{enwiki:972970373}}}

\newglossaryentry{organelle}{
	name={organelle},
	description={Any membrane-limited subcellular structure find in eukaroyic cells\cite{lodish2003molecular}}}
	
\newglossaryentry{plasma_membrane} {
	name={plasma membrane},
	description={The membrane surrounding a cell that separated the cell from its external environment; it consists of a phospholipid bilayer and associated membrane lipids and proteins\cite{lodish2003molecular}}}

\newglossaryentry{punctate} {
	name= {punctate},
	description={Marked with dots}}

\newglossaryentry{spatial_pyramid_pooling}{
	name={Spatial Pyramid Pooling},
	description={A pooling layer that removes the fixed-size constraint of the network, i.e. a CNN does not require a fixed-size input image. Specifically, we add an SPP layer on top of the last convolutional layer. The SPP layer pools the features and generates fixed-length outputs, which are then fed into the fully-connected layers (or other classifiers)\cite{he2015spatial}}}

\newglossaryentry{vesicle}{
	name= {vesicle},
	description={A vesicle is a structure within or outside a cell, consisting of liquid or cytoplasm enclosed by a lipid bilayer. Vesicles form naturally during the processes of secretion (exocytosis), uptake (endocytosis) and transport of materials within the plasma membrane. Alternatively, they may be prepared artificially, in which case they are called liposomes (not to be confused with lysosomes). If there is only one phospholipid bilayer, they are called unilamellar liposome vesicles; otherwise they are called multilamellar. The membrane enclosing the vesicle is also a lamellar phase, similar to that of the plasma membrane, and intracellular vesicles can fuse with the plasma membrane to release their contents outside the cell. Vesicles can also fuse with other organelles within the cell. A vesicle released from the cell is known as an extracellular vesicle\cite{enwiki:1000372338}}}
