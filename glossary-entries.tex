%   Copyright (C) 2021 Greenweaves Software Limited
%
%   This program is free software: you can redistribute it and/or
%   modify it under the terms of the GNU Lesser General Public
%   License as published by the Free Software Foundation; either
%   version 2.1 of the License, or (at your option) any later version.
%
%  This program is distributed in the hope that it will be useful,
%  but WITHOUT ANY WARRANTY; without even the implied warranty of
%  MERCHANTABILITY or FITNESS FOR A PARTICULAR PURPOSE.  See the
%  GNU  Lesser General Public License for more details
%
%  You should have received a copy of the GNU Lesser General Public License
%  along with this program.  If not, see <https://www.gnu.org/licenses/>.
%
%  To contact me, Simon Crase, email simon@greenweaves.nz

\newglossaryentry{actin_filaments}{
	name={Actin filaments},
	description={The major cytoskeletal protein of most cells is actin, which polymerizes to form actin filaments—thin, flexible fibers approximately 7 nm in diameter and up to several micrometers in length\cite{Cooper2000}}}


\newglossaryentry{aggresome} {
	name={Aggresome},
	description={In eukaryotic cells, an aggresome refers to an aggregation of misfolded proteins in the cell, formed when the protein degradation system of the cell is overwhelmed. Aggresome formation is a highly regulated process that possibly serves to organize misfolded proteins into a single location.}}


\newglossaryentry{centrosome} {
	name= {Centrosome},
	description={In cell biology, the centrosome is an organelle that serves as the main microtubule organizing center (MTOC) of the animal cell, as well as a regulator of cell-cycle progression. The centrosome is thought to have evolved only in the metazoan lineage of eukaryotic cells. Fungi and plants lack centrosomes and therefore use other structures to organize their microtubules. Although the centrosome has a key role in efficient mitosis in animal cells, it is not essential in certain fly and flatworm species.}}

\newglossaryentry{cytosol}{
	name= {Cytosol},
	description={The cytosol, also known as intracellular fluid (ICF) or cytoplasmic matrix, or groundplasm, is the liquid found inside cells. It is separated into compartments by membranes. For example, the mitochondrial matrix separates the mitochondrion into many compartments. }}


\newglossaryentry{endoplasmic_reticulum}{
	name= {Endoplasmic reticulum},
	description={The endoplasmic reticulum (ER) is, in essence, the transportation system of the eukaryotic cell, and has many other important functions such as protein folding. It is a type of organelle made up of two subunits – rough endoplasmic reticulum (RER), and smooth endoplasmic reticulum (SER). The endoplasmic reticulum is found in most eukaryotic cells and forms an interconnected network of flattened, membrane-enclosed sacs known as cisternae (in the RER), and tubular structures in the SER. The membranes of the ER are continuous with the outer nuclear membrane. The endoplasmic reticulum is not found in red blood cells, or spermatozoa. }}


\newglossaryentry{golgi_apparatus} {
	name={Golgi apparatus},
	description={A Golgi body, also known as a Golgi apparatus, is a cell organelle that helps process and package proteins and lipid molecules, especially proteins destined to be exported from the cell. Named after its discoverer, Camillo Golgi, the Golgi body appears as a series of stacked membranes.}}

\newglossaryentry{intermediate_filaments}{
	name= {Intermediate filaments},
	description={Intermediate filaments have a diameter of about 10 nm, which is intermediate between the diameters of the two other principal elements of the cytoskeleton, actin filaments (about 7 nm) and microtubules (about 25 nm). In contrast to actin filaments and microtubules, the intermediate filaments are not directly involved in cell movements. Instead, they appear to play basically a structural role by providing mechanical strength to cells and tissues.\cite{Cooper2000}}}

\newglossaryentry{microtubules} {
	name= {Microtubules} ,
	description={Microtubules are polymers of tubulin that form part of the cytoskeleton and provide structure and shape to eukaryotic cells. Microtubules can grow as long as 50 micrometres and are highly dynamic. The outer diameter of a microtubule is between 23 and 27 nm while the inner diameter is between 11 and 15 nm. They are formed by the polymerization of a dimer of two globular proteins, alpha and beta tubulin into protofilaments that can then associate laterally to form a hollow tube, the microtubule. The most common form of a microtubule consists of 13 protofilaments in the tubular arrangement.}}


\newglossaryentry{mitochondria} {
	name= {Mitochondria},
	description={Mitochondria are membrane-bound cell organelles (mitochondrion, singular) that generate most of the chemical energy needed to power the cell's biochemical reactions. Chemical energy produced by the mitochondria is stored in a small molecule called adenosine triphosphate (ATP). Mitochondria contain their own small chromosomes. Generally, mitochondria, and therefore mitochondrial DNA, are inherited only from the mother. }}


\newglossaryentry{mitotic_spindle} {
	name={Mitotic spindle},
	description={The mitotic spindle is a microtubule-made machine required for chromosome segregation during mitosis. Several pathways and molecules contribute to the assembly of this structure, which shows intrinsic dynamic properties. The dynamic nature of spindle microtubules plays a key role in error correction, as well as in the generation of force necessary to move chromosomes. Additionally, several molecular motors use spindle microtubules as tracks and, together with microtubule stabilizing as well as destabilizing proteins, they participate in proper spindle assembly and function. Here we summarize the essential concepts and principles behind the formation of this robust chromosome segregation machine.}}

\newglossaryentry{nuclear_bodies} {
	name= {Nuclear bodies},
	description={Nuclear bodies including nucleoli, Cajal bodies, nuclear speckles, Polycomb bodies, and paraspeckles are membrane-less subnuclear organelles. They are steady-state structures that dynamically respond to basic physiological processes as well as various forms of stress, altered metabolic conditions and alterations in cellular signaling. The formation of specific nuclear bodies has been suggested to follow stochastic and ordered assembly models. In addition, a seeding mechanism has been proposed to assemble, maintain, and regulate particular nuclear bodies. In coordination with noncoding RNAs, chromatin modifiers and other machineries, various nuclear bodies have been shown to sequester and modify proteins, process RNAs and assemble ribonucleoprotein complexes, as well as epigenetically regulate gene expression. Understanding the functional relationships between the three-dimensional organization of the genome and nuclear bodies is essential to fully uncover the regulation of gene expression and its implications in human diseases.\cite{mao2011biogenesis}}}

\newglossaryentry{nuclear_membrane}{
	name={Nuclear membrane},
	description={A nuclear membrane is a double membrane that encloses the cell nucleus. It serves to separate the chromosomes from the rest of the cell. The nuclear membrane includes an array of small holes or pores that permit the passage of certain materials, such as nucleic acids and proteins, between the nucleus and cytoplasm.\cite{segre2021}}}

\newglossaryentry{nuclear_speckles}{
	name={Nuclear speckles},
	description={Nuclear speckles, also known as interchromatin granule clusters, are nuclear domains enriched in pre-mRNA splicing factors, located in the interchromatin regions of the nucleoplasm of mammalian cells. When observed by immunofluorescence microscopy, they usually appear as 20–50 irregularly shaped structures that vary in size. Speckles are dynamic structures, and their constituents can exchange continuously with the nucleoplasm and other nuclear locations, including active transcription sites. Studies on the composition, structure, and dynamics of speckles have provided an important paradigm for understanding the functional organization of the nucleus and the dynamics of the gene expression machinery.\cite{spector2011nuclear}}}


\newglossaryentry{nucleoli_fibrillar_center}{
	name={Nucleoli fibrillar center},
	description={The nucleolus fibrillar center (FC) is a sub-compartment of most metazoan nucleoli. The transcription of ribosomal RNA (rRNA) genes generates 2 structures that are found in all nucleoli: the dense fibrillar component (DFC) and the granular component (GC). The DFC contains newly synthesized preribosomal RNA and a collection of proteins; the GC is made up of nearly completed preribosomal particles destined for the cytoplasm. In most metazoans, but generally not in lower eukaryotes, a third component, the FC, can be seen. The FC is surrounded by the DFC. The zone of transcription from multiple copy rRNA genes is in the border region between these 2 structures.}}


\newglossaryentry{nucleolus} {
	name= {Nucleolus},
	plural={Nucleoli},
	description={The nucleolus  is the largest structure in the nucleus of eukaryotic cells. It is best known as the site of ribosome biogenesis. Nucleoli also participate in the formation of signal recognition particles and play a role in the cell's response to stress. Nucleoli are made of proteins, DNA and RNA and form around specific chromosomal regions called nucleolar organizing regions. }}


\newglossaryentry{nucleoplasm} {
	name={Nucleoplasm},
	description={Similar to the cytoplasm of a cell, the nucleus contains nucleoplasm, also known as karyoplasm, or karyolymph or nucleus sap. The nucleoplasm is a type of protoplasm, and is enveloped by the nuclear envelope (also known as the nuclear membrane). The nucleoplasm includes the chromosomes and nucleolus. Many substances such as nucleotides (necessary for purposes such as DNA replication) and enzymes (which direct activities that take place in the nucleus) are dissolved in the nucleoplasm. The soluble, liquid portion of the nucleoplasm is called the nucleosol or nuclear hyaloplasm.}}


\newglossaryentry{plasma_membrane} {
	name={Plasma membrane},
	description={The plasma membrane, also called the cell membrane, is the membrane found in all cells that separates the interior of the cell from the outside environment. In bacterial and plant cells, a cell wall is attached to the plasma membrane on its outside surface. The plasma membrane consists of a lipid bilayer that is semipermeable. The plasma membrane regulates the transport of materials entering and exiting the cell.}}

\newglossaryentry{punctate} {
	name= {punctate},
	description={Marked with dots}}



\newglossaryentry{vesicles}{
	name= {Vesicles},
	description={In cell biology, a vesicle is a structure within or outside a cell, consisting of liquid or cytoplasm enclosed by a lipid bilayer. Vesicles form naturally during the processes of secretion (exocytosis), uptake (endocytosis) and transport of materials within the plasma membrane. Alternatively, they may be prepared artificially, in which case they are called liposomes (not to be confused with lysosomes). If there is only one phospholipid bilayer, they are called unilamellar liposome vesicles; otherwise they are called multilamellar. The membrane enclosing the vesicle is also a lamellar phase, similar to that of the plasma membrane, and intracellular vesicles can fuse with the plasma membrane to release their contents outside the cell. Vesicles can also fuse with other organelles within the cell. A vesicle released from the cell is known as an extracellular vesicle. }}



