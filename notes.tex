%   Copyright (C) 2021 Greenweaves Software Limited
%
%   This program is free software: you can redistribute it and/or
%   modify it under the terms of the GNU Lesser General Public
%   License as published by the Free Software Foundation; either
%   version 2.1 of the License, or (at your option) any later version.
%
%  This program is distributed in the hope that it will be useful,
%  but WITHOUT ANY WARRANTY; without even the implied warranty of
%  MERCHANTABILITY or FITNESS FOR A PARTICULAR PURPOSE.  See the
%  GNU  Lesser General Public License for more details
%
%  You should have received a copy of the GNU Lesser General Public License
%  along with this program.  If not, see <https://www.gnu.org/licenses/>.
%
%  To contact me, Simon Crase, email simon@greenweaves.nz


\documentclass[]{article}
\usepackage[acronym,toc]{glossaries}
\usepackage{url}
\usepackage{float}
\usepackage{caption}
\usepackage{subcaption}
\usepackage{graphicx}
\usepackage{amsmath}
\usepackage{amssymb}
\usepackage{tocloft}
\loadglsentries{glossary-entries}

\makeglossaries

%opening
\title{Human Protein Atlas - Single Cell Classification\\Notes}
\author{Simon Crase}

\begin{document}

\maketitle


\begin{abstract}
My notes on the Human Protein Atlas - Single Cell Classification project.
\end{abstract}
\tableofcontents
\listoftables
\section{The Task}

\begin{quotation}
	The labels are the organelles/structures in which the proteins are located. So you are correct in assuming that the labels only apply to the cells where green is present (as the green is the protein).
	
	If there are 4 cells and 3 have green staining and the image-level labels are Mitochondria, and Nucleoplasm.
	
	Cell 1:
	- Green looks to be in the Mitochondria. Therefore, the cell level is Mitochondria.
	
	Cell 2:
	- Green looks to be in the Nucleoplasm. Therefore, the cell level is Nucleoplasm.
	
	Cell 3:
	- Green looks to be in the Nucleoplasm and Mitochondria. Therefore, the cell level label is Nucleoplasm and Mitochondria
	
	Cell 4:
	- No green or green is not present in any organelle. Therefore, the cell level label is Negative. \cite{Schettler2021}
\end{quotation}

\begin{table}[H]
\begin{center}
	\caption{Labels}
	\begin{tabular}{|r|l|} \hline
			 0&\gls{nucleoplasm}\\ \hline
			 1&\gls{nuclear_membrane}\\ \hline
			 2&\glspl{nucleolus}\\ \hline
			 3&\gls{nucleoli_fibrillar_center}\\ \hline
			 4&\gls{nuclear_speckles}\\ \hline
			 5&\gls{nuclear_bodies}\\ \hline
			 6&\gls{endoplasmic_reticulum}\\ \hline
			 7&\gls{golgi_apparatus}\\ \hline
			 8&\gls{intermediate_filament}\\ \hline
			 9&\gls{actin} filaments\\\hline
			 10&\gls{microtubules}\\\hline
			 11&\gls{mitotic_spindle}\\\hline
			 12&\gls{centrosome}\\\hline
			 13&\gls{plasma_membrane}\\\hline
			 14&\gls{mitochondria}\\\hline
			 15&\gls{aggresome}\\\hline
			 16&\gls{cytosol}\\\hline
			 17&\gls{vesicles} and \gls{punctate} cytosolic patterns\\\hline
			 18&Negative \\ \hline
	\end{tabular}
\end{center}
\end{table}



\clearpage

\printglossary

\bibliographystyle{unsrt}
\addcontentsline{toc}{section}{Bibliography}
\bibliography{hpa-scc}
\end{document}
